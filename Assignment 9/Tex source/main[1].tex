\documentclass{article}
\usepackage[utf8]{inputenc}
\usepackage{tikz}
\usetikzlibrary{calc}
\usepackage{float}
\usepackage{graphicx}
\usepackage{listings}
\usepackage{karnaugh-map}
\title{Assignment 9 (GATE, EC2018,18)}
\author{SIKANDER KATHAT }
\date{25 December 2020}

\begin{document}

\maketitle

\section{ Question  18 MUX diagram }
\begin{figure}[!ht]
\centering
{
\input{figure3.tex}
}
\caption{question MUX diagram}
\label{MUX diagram)}
\end{figure}

\section{Question 18}
A 4:1 multiplexer is to be used for generating the output carry of a full adder. A and B are the bits to be added while C(in) is the input carry and Co/t is the output carry. A and B are to be used as select bits with A being more significant select bit.

Which one of the following statement correctly describes the choice of signals to be connected to the inputs I(0), I(1), I(2) and I(3) so that the output is Co/t?

\begin{enumerate}
    \item$I(0)=0,I(1)=C(in),I(2)=C(in),I(3)=1$
    \item$I(0)=1,I(1)=C(in),I(2)=C(in),I(3)=1$
    \item$I(0)=C(in),I(1)=0,I(2)=1,I(3)=C(in)$
    \item$I(0)=0,I(1)=C(in),I(2)=1,I(3)=C(in)$
\end{enumerate}

\section{Solution}
\begin{table}[!ht]
{

\begin{tabular}{|l|l|l|} \hline
A & B & C o/t  \\ \hline
0 & 0 & 0      \\
0 & 1 & C(in)  \\
1 & 0 & C(in)  \\
1 & 1 & 1      \\ \hline
\end{tabular}


}
\caption{TRUTH TABLE}
\label{table}
\end{table}


\begin{figure}[!ht]
\centering
{
\input{figure1.tex}
}
\caption{k-map for Sum(S)}
\label{kmap Sum(S)}
\end{figure}


    
    Boolean expression for Sum(S)-
    
    S=A$\overline{B}$ $\overline{C(in)}$ + $\overline{A}$ $\overline{B}$ C(in) + $\overline{A}$ B $\overline{C(in)}$ + ABC(in)
\newpage

\begin{figure}[!ht]
\centering
{
\input{figure2.tex}
}
\caption{k-map for Co/t}
\label{kmap Co/t}
\end{figure}
 
 Boolean expression for Co/t-

Co/t= BC(in) + AB + AC(in)

\begin{table}[!ht]
\centering
{
\documentclass{article}
\begin{document}
TRUTH TABLE-
\begin{table}[truth table]
\begin{tabular}{|l|l|l|l|l|}
\hline
A & B & C(in) & Sum(S) & Co/t \\ \hline
0 & 0 & 0     & 0      & 0    \\
0 & 0 & 1     & 1      & 0    \\
0 & 1 & 0     & 1      & 0    \\ 
0 & 1 & 1     & 0      & 1    \\
1 & 0 & 0     & 1      & 0    \\
1 & 0 & 1     & 0      & 1    \\
1 & 1 & 0     & 0      & 1    \\
1 & 1 & 1     & 1      & 1    \\ \hline
\end{tabular}
\end{table}

\end{document}


}
\caption{TRUTH TABLE for Co/t to be output}
\label{table 1 }
\end{table}
So, by the truth table for Co/t to be output,
we get 


{I(0)=0, I(1)= C(in), I(2)= C(in) and I(3)= 1}
\end{document};


